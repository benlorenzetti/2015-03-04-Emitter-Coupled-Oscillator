\documentclass[titlepage, letterpaper, 10.5pt]{article}
\usepackage[letterpaper, margin=1.25in]{geometry}
\usepackage{graphicx}
\usepackage{comment}
\usepackage{amsmath}
\usepackage{caption}
\usepackage{subcaption}
\usepackage{nicefrac}
\usepackage{tablefootnote}
\begin{document}

\title{Emitter-Coupled Oscillator}
\author{Author: Ben Lorenzetti}
\date{Project Start Date: February 26, 2015\\
Report Submission Date: March 23, 2015}
\maketitle

\clearpage
\mbox{}
\thispagestyle{empty}
\clearpage
\setcounter{page}{1}

\tableofcontents

\section{Objective}

To investigate the design and operation of an emitter-coupled astable multivibrator for 50
MHz operation.

\clearpage
\section{Principles of Operation}

In analog electronics, a sinusoidal oscillator is sometimes needed,
such as for power conversion, driving a motor, or for the carrier frequency in radio transmissions.
These can be built from an amplifier with a passive feedback network,
where the $\textrm{gain}>1$ and the feedback is positive at the frequency of operation
\footnote{the Barkhausen Criterion}.
The only other requirement is that the amplifer operates linearly.

In contrast, digital circuits usually require a square wave clock signal.
Ironically, a digital oscillator can be built starting from a basic digital component.

A flip-flop is a digital circuit with two stable states, made from two cross-connected amplifiers.
It is a fundamental component in digital logic.
In fact, when a master and slave flip-flop are connected in series they form a 1-bit register used in CPUs.
Usually, the cross-connection between the two amplifiers is resistive and one amplifier
will be driven in saturation while the other is in cutoff.
Amplifier in digital circuits are usually driven between saturation and cutoff,
making them unlinear, binary (two-state) devices.

DIGITAL FLIP-FLOP AND FLIP FLOP WITH REACTIVE EMITTER COUPLING

If the cross-connection is made to be reactive, then the flip-flop may not be stable in either state.
If it oscillates between the two states and produces a near-square wave, it is called a bistable multivibrator.
A capacitor is usually used for the cross-connection because it delays the instantaneous voltage change
that would normally occur when a transistor switches between saturation and cutoff.

Analog oscillators produce sinusoids; digital oscillators produce square waves.
An analog amplifer oscillates if it has reactive feedback that is positive;
a digital flip-flop oscillates if it has reactive cross-connection strong enough to switch states.
Analog oscillators use an amplifier in a linear range; digital oscillators use drive
amplifier to its two limits.
An analog oscillator may form unintentionally through parasitic feedback,
such as through the Miller capacitance of a BJT.
This happens to many frustrated students building amplifiers in Electronics Lab.
\footnote{see my last lab}
A digital osciallator may form unintentionally through propagation delay or carry-through
logic. This happens to many frustrated students simulating a CPU in Computer Architecture.
All of these comparisons between analog and digital oscillators can probably be gleaned
from the block, system-level diagrams in figure \ref{oscillator-block-diagrams}.

OSCILLATOR BLOCK DIAGRAMS AND CORRESPONDING V-T SIGNALS

The flip-flip circuit shown above was not really indended to oscillate.
It can be improved by replacing the emitter resistors with constant current mirrors
and adding voltage buffers in the cross-connections to improve charging time of the
switching junctions and improve output resistance.

ADVANCED CIRCUIT DIAGRAM

\end{document}
